% -*- root: ../paper.tex -*-

The \systemname knowledge-base is initially populated with a collection of example data sets.
This starting point produces low quality structure, requring careful curation of training data, as well as expert-provided heuristics.
This is intended to be an ongoing process, with continuous feedback from experts and users incrementally refining the knowledge-base.
In this section we outline the design of an interface that streamlines knowledge-base refinement, starting with a collection of random data
The central elements of this interface are 
(1) Visualizing the current quality of the knowledge base;
(2) Identifying problem name/match-quality function pairs;
(3) Refining records in the knowledge base by removing or merging existing data, or adding expert knowledge.
