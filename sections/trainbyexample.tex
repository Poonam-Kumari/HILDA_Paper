
To instantiate specific match-quality functions, we merge three sources of information: Learned Heuristics, Expert Augmentations, and User Feedback. 
The first category, learned heuristics, models the content and distribution of typical instances of columns with a similar name.
This distribution serves as a baseline match-quality function.
The second category, expert augmentations, modifies the first, increasing or decreasing values based on expert-provided descriptions of what should and should not appear in columns with this name.
The final category, user feedback, provides users with a way to confirm or override automated system choices, while also preserving these associations for future use.

\subsection{Learned Heuristics}
Learned heuristics serve as baseline match-quality functions by modeling typical instances of columns.
Specific modeling techniques vary by data type, so our first step was to assess what types exist.
We sampled a collection of \placeholder{\#\#\#} data sets from open data portals, as discussed later in Experiments (Section~\ref{sec:experiments}).
We then categorized the \placeholder{\#\#\#} columns in our sample into three broad types: 
(1) Numeric data, or any records consisting of digits, at most a single decimal point, and an optional exponent; 
(2) Enumerated types, based on an arbitrary threshold of 100 distinct values in the column; and 
(3) Textual data, or anything else.  
By far, the dominant type was numeric, so the preliminary efforts we outline in this paper focus on modeling numeric data.

\tinysection{Numeric Data}
We considered a range of options for modeling numeric data and settled on an approach based on numerical distributions.
In comparison to more complex approaches like neural networks, this approach is simple, efficient, and well understood.
Simply put, given a number of example column instances, we explore a range of numerical distributions and select the one with the best fit.
Our preliminary implementation of \systemname explores three different distributions: Uniform, Normal, and \placeholder{Zipfian}\footnote{As we show in the experiments section, these three distributions are sufficient to describe the vast majority of data in our sample set.}.  
For each distribution, we find the parameters ($\ell,h,\mu,\sigma,a$) that minimize the root-mean-squared (RMS) error between the predicted and observed frequencies. 
\todo{Gourab: We are binning only for distribution fitting, no smoothing}
$$\mathbb U(\ell, h)\;\;\;\;\;\mathbb N(\mu, \sigma)\;\;\;\;\;\mathbb Z(a)$$
We then select the one model with the lowest overall RMS error.

The resulting match-quality function is the probability of the column values being a representative sample drawn from this distribution.
We measure this probability by the \placeholder{RMS Error} between the distribution of the sampled values and the distribution modeling the name.